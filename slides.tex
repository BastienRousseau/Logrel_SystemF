\documentclass[xcolor={dvipsnames}]{beamer}
\title{Formalizing logical relation for System F type safety in Coq}
\author{Bastien ROUSSEAU}
\date{2023}

\usepackage[utf8]{inputenc}
\usepackage{amsmath}
\usepackage{amssymb}
\usepackage{stmaryrd}
\usepackage{mathpartir}
\usepackage{pftools}

% \usepackage{capt-of}
% \usepackage{generalMacros}
% \usepackage[usenames,dvipsnames]{xcolor}
% \usepackage{circledsteps}
% \usepackage{syntaxColor}
% \newcommand{\lrp}[2]{\llbracket #2 \rrbracket_{#1}}
\newcommand{\lr}[3]{\llbracket #2 \rrbracket_{#1}(#3)}
\newcommand{\lrv}[2]{\lr{#1}{#2}{v}}
\newcommand{\typed}[3]{#1 \vdash #2 : #3}
\newcommand{\hstep}{\rightsquigarrow}
\newcommand{\step}{\rightarrow}
\newcommand{\mstep}{\step^{\ast}}

\let\originalleft\left
\let\originalright\right
\renewcommand{\left}{\mathopen{}\mathclose\bgroup\originalleft}
\renewcommand{\right}{\aftergroup\egroup\originalright}


\newcommand{\fst}[1]{\com{fst}~#1}
\newcommand{\snd}[1]{\com{snd}~#1}
\renewcommand{\tt}{\com{tt}}
\newcommand{\lam}[2]{\lambda #1.~#2}
\newcommand{\app}[2]{#1~#2}
\newcommand{\tlam}[1]{\Lambda#1}
\newcommand{\tapp}[1]{#1~\_}
\newcommand{\pair}[2]{\left(#1,\,#2 \right)}
\renewcommand{\ctx}{\ensuremath{\mr{K}}}
\newcommand{\empctx}{\bullet}
\newcommand{\subst}[3]{#1.\left[#2/#3\right]}
\newcommand{\shift}[3]{\uparrow^{#2}_{#3}(#1)}


\newcommand{\tyunit}{\mr{unit}}
\newcommand{\ty}{\tau}
\newcommand{\typair}[2]{#1 \times #2}
\newcommand{\tyarrow}[2]{#1 \rightarrow #2}
\newcommand{\tyforall}[2]{\forall #1.~#2}

\newcommand{\exprctx}{\Gamma}
\newcommand{\tyctx}{\Delta}

\newcommand{\safe}{\ensuremath{\mr{Safe}}}
\newcommand{\norm}{\ensuremath{\mr{Norm}}}
\newcommand{\val}{\ensuremath{\mr{Val}}}

\newcommand{\sctx}{\xi}
\newcommand{\sfun}{\gamma}

\newcommand{\systemF}{System F}


% \theoremstyle{plain}
% \newtheorem*{theorem*}{Theorem}
% \newtheorem*{lemma*}{Lemma}
% \newtheorem*{definition*}{Definition}

% \usepackage{biblatex}
% \addbibresource{/home/au711415/Work/SystemF_Safe/biblio.bib}
% \addbibresource{~/Work/master.bib}


% \usepackage{iris}
% \usepackage{macrosIris}
% \usepackage{macrosCerise}


% \usepackage{listings}
% \usepackage{setspace}
% \usepackage{subcaption}
% \usepackage{wrapfig}
% \usepackage{rotating}
% \usepackage[normalem]{ulem}

% \captionsetup[subfigure]{labelformat=empty}
% % Adjust the size of tabulars
% \usepackage{adjustbox}
% % Color tabular cells
% \usepackage{xcolor,colortbl}
% % Color verbatim
% \usepackage{fancyvrb}
% % Pretty tabulars
% \usepackage{booktabs}
% \usepackage{subcaption}

% % Better tabular with fixed-size columns
% \usepackage{tabularx}
% \makeatletter
% \newcommand\cellwidth{\TX@col@width}
% \newcolumntype{C}{>{\centering\arraybackslash}X}
% \newcolumntype{Z}{>{\raggedleft\arraybackslash}X}
% \makeatother

% MOTIVATION FOR STRONG NORMALIZATION

% Theme
\usecolortheme{beaver}
\beamertemplatenavigationsymbolsempty
\setbeamercolor{palette primary}{bg=white,fg=black}
\setbeamercolor{palette secondary}{bg=white,fg=black}
\setbeamercolor{palette tertiary}{bg=white,fg=black}
\setbeamercolor{palette quaternary}{bg=white,fg=black}
\setbeamercolor{structure}{bg=white,fg=black}
\setbeamercolor{titlelike}{bg=white,fg=black}
\setbeamercolor{title}{bg=white,fg=black}
\setbeamercolor{title in head/foot}{bg=white,fg=black}
\setbeamercolor{section in head/foot}{bg=white,fg=black}
\setbeamercolor{subsection in head/foot}{bg=white,fg=black}
\setbeamercolor{frametitle}{bg=white,fg=black}
\setbeamertemplate{footline}[frame number]
\newcommand\xxsectiontitle[1]{\begin{center}\Huge{#1}\end{center}}
\renewcommand\section[1]{\begin{frame}[noframenumbering]{}\xxsectiontitle{#1}\end{frame}}
\setbeamertemplate{blocks}[rounded][shadow=true]
\setbeamercolor{block title}{bg=black!5, fg=black}


\begin{document}

\frame{\titlepage}

\begin{frame}
  \frametitle{Introduction}

  \begin{block}{Block}
    \begin{itemize}
      \item item
      \item item
    \end{itemize}
  \end{block}

  \begin{block}{Block}
    \begin{itemize}
      \item item
      \item item
    \end{itemize}
  \end{block}

\end{frame}

\begin{frame}{Outline}
  \begin{enumerate}
    \item On paper
    \item Implementation
          \begin{itemize}
            \item Binders
            \item Generalization
          \end{itemize}
    \item Extensions
  \end{enumerate}
\end{frame}

\section{System F}%

\begin{frame}
  \frametitle{System F (variant) --- Syntax }

  \begin{block}{Types}
    $\tau ::= \mathrm{bool}\ |\ \tau \rightarrow \tau$
  \end{block}

  \begin{block}{Terms}
    $e ::= x\ |\ \mathrm{true}\ |\ \mathrm{false}\
    |\ \text{if } e \text{ then } e \text{ else } e\ |\ \lambda x : \tau.\ e\ |\ e\ e$
  \end{block}

  \begin{block}{Values}
    $v ::= x\ |\ \mathrm{true}\ |\ \mathrm{false}\ |\ \lambda x : \tau.\ e$
  \end{block}
\end{frame}

\begin{frame}
  \frametitle{System F (variant) --- Evaluation Call-By-Value }

  \begin{block}{Evaluation context}
    $E ::= []\ |\ \text{if } E \text{ then } e \text{ else } e\ |\ E\ e\ |\ v\ E$
  \end{block}
  \begin{block}{Evaluation reduction}
    \vspace{1em}
    \begin{mathparpagebreakable}
      \inferH
      {E-IfTrue}
      {  }
      {  \text{if } \mathrm{true} \text{ then } e_{1} \text{ else } e_{2} \rightarrow e_{1}}
      \and

      \inferH
      {E-IfFalse}
      {  }
      {  \text{if } \mathrm{false} \text{ then } e_{1} \text{ else } e_{2} \rightarrow e_{2}}
      \and

      \inferH
      {E-App}
      {  }
      {  (\lambda x : \tau .\ e)\ v \rightarrow e [v / x]}
      \and

      \inferH
      {E-Step}
      {e \rightarrow e'}
      {E[e] \rightarrow E[e']}

    \end{mathparpagebreakable}
    \vspace{-1.5em}
  \end{block}
\end{frame}


\begin{frame}
  \frametitle{System F (variant) --- Typing }
  % Typing judgement
  \begin{block}{Typing judgement}
    \vspace{1em}
    \begin{mathparpagebreakable}
      \inferH
      {T-True}
      {  }
      {\Gamma \vdash \mathrm{true} : \mathrm{bool}}
      \and

      \inferH
      {T-False}
      {  }
      {\Gamma \vdash \mathrm{false} : \mathrm{bool}}
      \and

      \inferH
      {T-Var}
      {\Gamma (x) = \tau}
      {\Gamma \vdash x : \tau}
      \and

      \inferH
      {T-Abs}
      {\Gamma, x: \tau_{1} \vdash e : \tau_{2}}
      {\Gamma \vdash \lambda x: \tau_{1}. e : \tau_{1} \rightarrow \tau_{2}}
      \and

      \inferH
      {T-App}
      {\Gamma \vdash e_{1} : \tau_{1} \rightarrow \tau_{2} \\ \Gamma \vdash e_{2} : \tau_{1}}
      {\Gamma \vdash e_{1}\ e_{2} : \tau_{2}}
      \and

      \inferH
      {T-IfThenElse}
      {\Gamma \vdash e : \mathrm{bool}
        \\ \Gamma \vdash e_{1} : \tau
        \\ \Gamma \vdash e_{2} : \tau}
      {\Gamma \vdash \text{if } e \text{ then } e_{1} \text{ else } e_{2} : \tau}
    \end{mathparpagebreakable}
    \vspace{-1.5em}
  \end{block}
\end{frame}

\section{Implementation}%

\begin{frame}
  \frametitle{De Bruijn binders}
  \begin{block}{Definitions}
    \begin{itemize}
      \item determinism $\Rightarrow$ strongly normalizing and normalizing equivalent
    \end{itemize}
  \end{block}
\end{frame}


\begin{frame}
  \frametitle{Substitution lemma}
  \begin{center}
    center
  \end{center}

  % Show the case that does not work T-APP and explain why
\end{frame}

\section{Extensions}%

\begin{frame}
  \frametitle{Iris}
  Iris

  \begin{block}{Recursive types}
    \begin{itemize}
      \item $\tau ::= \ldots\ |\ \mu \alpha. \tau$
      \item $\mu$ is a fixpoint operator
      \item Breaks normalization
    \end{itemize}
  \end{block}

  \begin{block}{Reference types}
    \begin{itemize}
      \item $\tau ::= \ldots\ |\ \text{ref } \tau$
      \item Landin's knot: encode the recursion through the heap
      \item Breaks normalization
    \end{itemize}
  \end{block}
\end{frame}

\begin{frame}
  \frametitle{Properties}
  \begin{block}{Normalize}
  \end{block}

  \begin{block}{Contextual equivalence}
  \end{block}
\end{frame}

\begin{frame}
  \frametitle{Conclusion}
\end{frame}

\end{document}
